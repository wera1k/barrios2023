\documentclass{article}
\begin{document} 
	
\section{Sample Document and Key Concepts}

This is a short document to illustrate tha basic use of Latex.
Simply leave a blank line to get a new paragraph; indentation is automatic.

Mathematical expressions such as $y = 3 \ sin x$ are obtained with dollar signs.

Equations can be displayed, as in
\[
	y = 3 \sin x
\]
Numbered equations are also possible:
\begin{equation}\label{equa}
	y = 3 \sin x
\end{equation}	
Because we have labeled this equation we can refer to it without having to know its number. thus, the preceding equation was
number~(\ref{equa}).

Powers (superscripts), as in $x^2$, are obtained with \verb"^"; more complicated power must live in curly braces: $x^{2+\alpha}$

Likewise, subscripts are obtained woht the underscore: $y_3$ or $y_{n+1}$.

We can get both with $x_{n+1}^{2+\alpha}$.

The special characters \&, \$, \%, \_, \{, \}, and \# may be printed by preceding each with a backslash. We can then put text in \{curly braces\}.

It is likely that 50\% of the time you will be frustrated because you forgot to precede the \% symbol by a backslash.
% sin el slash sirve para comentar. Si lo pongo sin el slash, lo que escriba después no aparecerá en el texto ya que se toma como comentario

\section{Type Style}

\textup{Upright type}
\ \\ % esto es un enter
\textit{Italic type}
\ \\
\textsl{Slanted type}
\ \\
\textsc{Small caps type}
\ \\ 
\textmd{Medium} \textbf{Boldface}
\ \\
\textrm{Roman} \textsf{Sans serif} \texttt{Typewriter}
\ \\
\textsl{Don't \textbf{overuse} type-changing.}
\textsf{It \textit{annoys} the \textsc{reader}.}
\texttt{And loses \textsl{impact}.}
\ \\
\textsc{Pile on \emph{lots} of subtlety.}
\textsf{Sans serif adds a little  \emph{je ne sais rein}.}
\textsl{Nouns should \emph{never} be verbed.}
\ \\ \ \\
{\LARGE LARGE text} makes ideal
{\Large Large text} for
a shortsighted people;
{\tiny tiny text} makes
ideal {\scriptsize scriptsize text} for longsighted people.

\section{Enviroments}

\begin{itemize}
	\item Every sentence should make sense in isolation. 
	Like that one.
	\item There is a lot to be said for brevity.
	\item Many words can ostensibly be deleted.
	\item Eschew the highfalutin.
	\item Understatement is a mindblowingly effective weapon.
\end{itemize}	
\ \\	
	
\begin{enumerate}
	\item Spellcheckers are not perfect; they can kiss may errs.
	\item Somebody once said that all quotes should neaccurately attributed.
	\item The importance of comprehensive cross-referencing will be covered elsewhere.
\end{enumerate}	
\ \\

\begin{description}
		\item[Rule 1.] Mixed metaphors can kill two birds without a paddle.
		\item[Rule 2.] Similes are about as much use as a chocolate teapot.	 
\end{description}
\ \\

%Listas anidadas
\begin{enumerate}
	\item Punctuation
	\begin{enumerate}
		\item Don't use commas, to separate text unnecessarily.
		\item Avoid ugly abrv's'ns.
	\end{enumerate}
	\item Spelling
	\begin{enumerate}
		\item If there's a particular word ya can never spell, use a pnemonic.
		\item Take care with pluri.
	\end{enumerate}	
\end{enumerate}
\ \\ \ \\ \ \\ \ \\ \ \\ \ \\ \ \\ \ \\ \ \\ \ \\ \ \\ \ \\ \ \\ \ \\ \ \\

\begin{center}
	{\large\textbf{Assigment 1}}\\
	Sue d'Onym\\
	MS601
\end{center}	
\ \\

The marks for the 1996 class are more respectable. \ \\
\begin{tabular}{lrc}
	Name & Mark & Grade \\
	\hline
	Emma Winner & 99 & A+ \\
	Scott Passmark & 51 & C \\
	Shirley Knott & 5 & F 
\end{tabular} 
\ \\	
The avarage mark is well over 50\%
\ \\

\begin{center}
	\begin{tabular}{|l||r|c|}
		\hline
		Name & Mark & Grade \\
		\hline \hline
		Emma Winner & 99 & A+ \\
		Scott Passmark & 51 & C \\
		Shirley Knott & 5 & F \\
		\hline
	\end{tabular}
\end{center}
\ \\

\begin{tabular}{|l||r|r|}
	\hline
	& \multicolumn{2}{c|}{Marks}\\
		\cline{2-3}
	Name & MS601 & MS602 \\
	\hline \hline
		Emma Winner & 99 & 51 \\
		Scott Passmark & 51 & 50 \\
		Shirley Knott & 5 & 49 \\
		\hline	
\end{tabular}	
\ \\ \ \\




The results given in Table~\ref{tab:a} show the very satisfactory performance off the 1996 class, whose avarage is over 50\% (Note that we have referred to the number of the table before it appears.)
\begin{table}
	\begin{center}
		\begin{tabular}{lrc}\hline
			Name & Mark & Grade \\
			\hline
			Emma Winner & 99 & A+ \\
			Scott Passmark & 51 & C \\
			Shirley Knott & 5 & F \\
	\end{tabular}	
	\caption{Class Mark List}\label{tab:a}
	\end{center}
\end{table}

abc fbgfbgfbgf \cite{Stigler-2005} jhgjhgjhgjhgjyg


	\nocite{*}
	\bibliography{../Referencias/REFERENCIASejemplo}
	\bibliographystyle{plain}	

\end{document}