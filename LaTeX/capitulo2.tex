\documentclass{book}
\usepackage[utf8]{inputenc}
\usepackage{epsfig}
\usepackage{graphicx}
%\usepackage[dvips]{graphics}
\usepackage[spanish]{babel}
\usepackage[paperwidth=17cm, paperheight=22.5cm, bottom=2.5cm, right=2.5cm]{geometry}
\usepackage{array}
\usepackage{multirow}
	
\begin{document}	

	\chapter{Introducción}  
	Descripción general del tema y de cómo se tratará todo. Hacer al final
	
	\chapter{Análisis descriptivo} 
		\begin{itemize}
			\item Poner de qué se trata, los datos de medicina, las imágenes de medicina, y contexto
			
			Los datos usados en la tesis son datos reales, no se incluyen datos personales de los pacientes, más que su edad, para conservar la privacidad de estos. 
			
			%El problema central es el siguiente: 
			
			%En este proyecto había cuatro radiólogos leyendo imágenes de resonancia magnética de caderas, usando la secuencia estándar (SOC = standard of care) y una nueva llamada IDEAL. Los radiólogos evalúan el daño en el cartílago en la cabeza del fémur usando la "carátula del reloj" para descubrir su localización. A los pacientes con "labral tears" (desgarres) son operados con artroscopía, y el cirujano evalúa el daño "real". La pregunta era si IDEAL era mejor que SOC para visualizar el daño, y evaluar que tan repetibles son las lecturas entre radiólogos y las del mismo radiólogo con las dos secuencias.	
			
			Antes de entrar al análisis, se dará una breve explicación de lo que son las roturas del labrum acetabular, para así comprender mejor los resultados.
			
	        \section{Contexto médico}
	        La articulación de la cadera está formada por la cabeza femoral (superficie convexa o bola) y por el acetábulo (cavidad articular). El labrum acetabular es un borde de tejido blando, o fibrocartílago, que rodea el acetábulo. El labrum ayuda a dar estabilidad a la cadera y a proteger la unión entre la cabeza del fémur y el acetábulo. (Fuente: US San Diego Health) 
	        
	        % Imagen de la anatomía de la cadera. Poner fuente
	      
	        
	        El labrum puede sufrir una ruptura debido a lesiones o degeneración.  Este tipo de lesiones son comunes en atletas que practican fútbol, fútbol americano, ballet, gimnasia, hockey y golf, entre otros. (Fuente: MayoClinic)
	        
	        %Imagen de tear. Poner fuente
	      
	      
	        Para diagnosticar una lesión se puede hacer uso de radiografías, pero para obtener mayor información se usa la Resonancia Magnética (RM). En caso de que el paciente necesite intervención quirúrgica se recurre a la artroscopía de cadera, que es un procedimiento cuyo objetivo el reinsertar el labrum roto y reparar cualquier anomalía ósea que pueda tener la cadera (Fuente: Clínica Meds). --- La lectura de la RM se hace en términos de horas de un reloj de manecillas, por ejemplo, 12 a 3. ---
	        
			% Imágenes de RM y labrum (reparado). Poner fuente
			
		\end{itemize}
		
		\begin{table}
			\begin{tabular}{|l|l|l|l|l|l|l|}
				\hline
				\multirow{2}{*}{Dataset} &
				\multicolumn{2}{c}{A} &
				\multicolumn{2}{c}{B} &
				\multicolumn{2}{c|}{C} \\
				& O.B.R & A.R & O.B.R & A.R & O.B.R & A.R \\
				\hline
				D1 & 2.1\% & 2.1\% & 2.1\% & 2.1\% & 2.1\% & 2.1\% \\
				\hline
				D2 & 11.6\% & 11.6\% & 11.6\% & 11.6\% & 11.6\% & 11.6\% \\
				\hline
				D3 & 5.5\% & 5.5\% & 5.5\% & 5.5\% & 5.5\% & 5.5\% \\
				\hline
			\end{tabular}
		\end{table}
		
    	
		\begin{table}[h]
		\centering
		\scalebox{.8}{	
			\begin{tabular}{|c||c|c|c|c|c|c|c|c|c|}
				\hline 
				\multirow{2}{*}{ID} &
				\multirow{2}{*}{Cirujano} &
					\multicolumn{2}{c|}{Radiólogo 1}   &  
					\multicolumn{2}{c|}{Radiólogo 2} &
					\multicolumn{2}{c|}{Radiólogo 3} &
					\multicolumn{2}{c|}{Radiólogo 4} 
					\\ \cline{3-10}
				&  &SOC & IDEAL & SOC & IDEAL & SOC & IDEAL & SOC & IDEAL \\
				\hline 
				1 & 12 a 3 & 12 a 3 & 12 a 3 & 1 a 3 & 1 a 3 & 1 a 2 & 1 a 2 & 12 a 230 & 12 a 230 \\
				\hline		
			\end{tabular}
		}
		\end{table}
		
		
		\begin{table}
		\centering
		\scalebox{.9}{
			\begin{tabular}{|c||c|c|c|c|c|c|c|c|c|c|c|c|c|c|c|c|c|c|} 
				\hline
				\multirow{3}{*}{ID} &
				\multicolumn{2}{c|}{\multirow{2}{*}{Cirujano}} & 
				\multicolumn{4}{c|}{Radiólogo 1} &
				\multicolumn{4}{c|}{Radiólogo 2} &
				\multicolumn{4}{c|}{Radiólogo 3} &
				\multicolumn{4}{c|}{Radiólogo 4} 
				\\ \cline{4-19} 
				  & %espacio para id
				\multicolumn{2}{c|}{} & %columnas para cirujano
				\multicolumn{2}{c|}{SOC} & \multicolumn{2}{c|}{IDEAL} & %columnas para Rad 1
				\multicolumn{2}{c|}{SOC} & \multicolumn{2}{c|}{IDEAL} & %columnas para Rad 2
				\multicolumn{2}{c|}{SOC} & \multicolumn{2}{c|}{IDEAL} & %columnas para Rad 3
				\multicolumn{2}{c|}{SOC} & \multicolumn{2}{c|}{IDEAL}  %columnas para Rad 4
				\\ \cline{2-19} 
				  & %espacio para id
				X & Y & %cirujano
				X & Y & %rad1 soc
				X & Y & %rad1 ideal
				X & Y & %rad2 soc
				X & Y & %rad2 ideal
				X & Y & %rad3 soc
				X & Y & %rad3 ideal
				X & Y & %rad4 soc
				X & Y  %rad4 ideal
				\\ \hline 
		        1 & 
		        12 & 3 & %cirujano
		        12 & 3 & 12 & 3 & %rad 1
		        1 & 3 & 1 & 3 & %rad 2
		        1 & 2 & 1 & 2 & %rad3
		        12 & 230 & 12 & 230	%rad4
		        \\ \hline				
			\end{tabular}
			}
		\end{table}
		
	
		\begin{table}
			\centering
			\scalebox{.75}{
				\begin{tabular}{|c||c|c|c|c|c|c|c|c|c|c|c|c|c|c|c|c|c|c|} 
					\hline
					\multirow{2}{*}{ID} &
					\multicolumn{2}{c|}{\multirow{3}{*}{Cirujano}} & 
					\multicolumn{4}{c|}{Radiólogo 1} &
					\multicolumn{4}{c|}{Radiólogo 2} &
					\multicolumn{4}{c|}{Radiólogo 3} &
					\multicolumn{4}{c|}{Radiólogo 4} 
					\\ \cline{4-19} 
					& %espacio para id
					\multicolumn{2}{c|}{} & %columnas para cirujano
					\multicolumn{2}{c|}{SOC} & \multicolumn{2}{c|}{IDEAL} & %columnas para Rad 1
					\multicolumn{2}{c|}{SOC} & \multicolumn{2}{c|}{IDEAL} & %columnas para Rad 2
					\multicolumn{2}{c|}{SOC} & \multicolumn{2}{c|}{IDEAL} & %columnas para Rad 3
					\multicolumn{2}{c|}{SOC} & \multicolumn{2}{c|}{IDEAL}  %columnas para Rad 4
					\\ \cline{1-19} 
					&  %espacio para id
					12 & 3 & %cirujano
					12 & 3 & 12 & 3 & %rad 1
					1 & 3 & 1 & 3 & %rad 2
					1 & 2 & 1 & 2 & %rad3
					12 & 230 & 12 & 230	%rad4
					\\ \cline{2-19} 
					1 & 
					X & Y & %cirujano
					X & Y & %rad1 soc
					X & Y & %rad1 ideal
					X & Y & %rad2 soc
					X & Y & %rad2 ideal
					X & Y & %rad3 soc
					X & Y & %rad3 ideal
					X & Y & %rad4 soc
					X & Y  %rad4 ideal					
					\\ \cline{2-19}
					&
					0 & 1.57 &
					0 & 1.57 &
					0 & 1.57 &
					0.52 & 1.04 &
					0.52 & 1.04 &
					0.52 & 0.52 &
					0.52 & 0.52 &
					0 & 1.30 &
					0 & 1.30
					\\ \hline						
				\end{tabular}
			}
		\end{table}

	
	\begin{table}[]
		\centering
		\caption{My caption}
		\label{my-label}
		\begin{tabular}{|c|c|c|c|c|c|c|c|c|c|c|c|c|c|c|c|c|c|c|}
			\hline
			\multirow{3}{*}{} & 
			\multicolumn{2}{c|}{\multirow{2}{*}{}} & 
			\multicolumn{4}{c|}{}  & 
			\multicolumn{4}{c|}{}  & 
			\multicolumn{4}{c|}{}  & 
			\multicolumn{4}{c|}{}                         
			\\ \cline{4-19} & 
			 \multicolumn{2}{c|}{} &                
			 \multicolumn{2}{c|}{} & 
			 \multicolumn{2}{c|}{} & 
			 \multicolumn{2}{c|}{} & 
			 \multicolumn{2}{c|}{} & 
			 \multicolumn{2}{c|}{} & 
			 \multicolumn{2}{c|}{} & 
			 \multicolumn{2}{c|}{} & 
			 \multicolumn{2}{c|}{} 
			 \\ \cline{2-19} 
			&                    &                   &           &           &           &           &           &           &           &           &           &           &           &           &           &           &           &           \\ \hline
			&                    &                   &           &           &           &           &           &           &           &           &           &           &           &           &           &           &           &           \\ \hline
		\end{tabular}
	\end{table}
		
		
		
		\begin{table}
		\begin{tabular}{|c|c|c|c|c|c|} 
			\hline
			\multirow{2}{*}{A} & % id
			\multicolumn{2}{c|}{User B} & %
			\multicolumn{2}{c|}{User C} & \multirow{3}{*}{D}\\
			\cline{2-5}
			& \multicolumn{2}{c|}{Value} & \multicolumn{2}{c|}{Value} & \\
			\cline{2-5}
			& B1 & B2 & C1 & C2 & \\
			\hline
			& & & & & \\
			\hline
			& & & & & \\
			\hline
			% etc. ...
		\end{tabular}
		\end{table}
		
		
		
		\begin{table}[h]
			\centering
			{\renewcommand{\arraystretch}{1.5}	
				\begin{tabular}{|c||c|c|}
					\hline 
                    \multicolumn{1}{|c||}{Reloj} &
					\multicolumn{2}{c|}{Radianes} \\    
					\hline	
					12 & 0 & 0 \\ \hline
					1230 & $\frac{1\pi}{12}$ & 0.2618 \\ \hline
					1 & $\frac{2\pi}{12}$ & 0.5236 \\ \hline 
					130 & $\frac{3\pi}{12}$ & 0.7854 \\ \hline
					2 & $\frac{4\pi}{12}$ & 1.0472 \\ \hline
					230 & $\frac{5\pi}{12}$ & 1.3090 \\ \hline
					3 & $\frac{6\pi}{12}$ & 1.5708 \\ \hline
					330 & $\frac{7\pi}{12}$ & 1.8326 \\ \hline
					4 & $\frac{8\pi}{12}$ & 2.0944 \\ \hline
					430 & $\frac{9\pi}{12}$ & 2.3562 \\ \hline
					5 & $\frac{10\pi}{12}$ & 2.6180 \\ \hline
					530 & $\frac{11\pi}{12}$ & 2.8798 \\ \hline
					6 & $\frac{12\pi}{12}$ & 3.1416 \\ \hline
					630 & $\frac{13\pi}{12}$ & 3.4034 \\ \hline
					7 & $\frac{14\pi}{12}$ & 3.6652 \\ \hline
					730 & $\frac{15\pi}{12}$ & 3.9270 \\ \hline
					8 & $\frac{16\pi}{12}$ & 4.1889 \\ \hline
					830 & $\frac{17\pi}{12}$ & 4.4506 \\ \hline
					9 & $\frac{18\pi}{12}$ & 4.7124 \\ \hline
					930 & $\frac{19\pi}{12}$ & 4.9749 \\ \hline
					10 & $\frac{20\pi}{12}$ & 5.2360 \\ \hline
					1030 & $\frac{21\pi}{12}$ & 5.4978 \\ \hline
					11 & $\frac{22\pi}{12}$ & 5.7596 \\ \hline
					1130 & $\frac{23\pi}{12}$ & 6.0214 \\ \hline
				\end{tabular}
			}\quad
		\end{table}		
		
		
		
	
		
		
		
		
\end{document}
			
			